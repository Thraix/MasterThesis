\chapter{Conclusion}\label{cha:conclusion}
This chapter answers the research questions by summarizing the results and discussions.
It also gives a conclusive answer as to which algorithm one should use when implementing an optimal scanline algorithm.

\section{Research Questions}
The research questions presented in the introduction are reiterated and answered in the following paragraphs.

\begin{enumerate}
\item Which line voxelization algorithm performs best for the optimal scanline? 
\end{enumerate}
For all models and voxel grid resolution, the floating-point version, RLV, performed better.
ILV performed anywhere between 20-250\% worse compared to RLV.
Bresenham performed anywhere between 20-500\% worse compared to RLV.
Interestingly, ILV performed better than Bresenham in every case but two.
Although, the difference was not as dramatic as between RLV and ILV, except for a few outliers which performed way worse.

The reason for the big difference between the floating-point and integer versions was due to the complexity of finding the scanline endpoints.
For RLV, it was simply a case of increasing the length of the scanline direction and performing a reverse projection of the direction to the triangle's edges.
For ILV and Bresenham, finding the endpoints required iterating through the edge voxelizations and performing a boundary test based on the previous scanline and the current voxel.

The reason for ILV being faster than Bresenham had less to do with Bresenham being worse in general.
It had more to do with Bresenham not being able to step the line voxelization by one voxel each iteration.
This in turn required extra overhead to overcome.

\begin{enumerate}
\setcounter{enumi}{1}
\item How great is the approximation error of the integer versions of the optimal scanline?
\end{enumerate}
The error for the integer versions of the optimal scanline was determined to be around 20-25\%.
This was based on the Jaccard distance between RLV and ILV/Bresenham.
One outlier of the data was the dragon model, with 871414 triangles, where the error started at around 9\% for a voxel grid resolution of 128.
It however gradually grew as the resolution increased.
The reason was likely due to the triangles of the model being mostly within a single or a few voxels.
Making it less likely to cause an error.

There also turned out to be a minor error ($\sim$5\%) between ILV and Bresenham, even though they both used the same underlying algorithm.
This was caused due to cases where the scanline passes through multiple voxels, such that the choice of voxel was ambiguous.
An example of this was shown in \figref{fig:ilv-bresen-error}.

\section{Choice of Algorithm}
From the results given there is only one obvious choice as to which algorithm one should use.
This is of course the floating-point version using RLV.
The algorithm had the best performance in terms of runtime.
It is also closer to the ground truth, as it only voxelizes voxels touching the triangle.
Due to approximation errors, this is not the case for the integer versions.
Given those results, there should be no reason not to choose RLV.
