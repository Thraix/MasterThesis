\chapter{Introduction}\label{cha:intro}
This chapter gives the reader an introduction to voxels, its uses and ways to create voxels from a model.
It also serves the purpose of motivating the thesis, as well as describing the problem it is out to solve.
Finally, it defines the delimitations of the thesis and presents the company, at which this thesis was conducted at.

\section{Background}
In computer graphics, a 3D model is built up of triangles which describes the surface of the model.
This means it does not store any information to represent the inside of the model.
In order to represent the inside, a volumetric data structure is needed.
Usually, this is done by storing \textit{voxels} in a uniform 3D grid.
A voxel can be described as data at a position in a grid.
This can be any form of data, such as occupancy, color, material or density.

% \textit{Voxel} is a word that has a lot of meaning in computer graphics.
% The simplest form of a voxel is a cube in a 3D-environment, but other representations of voxel data also exist.
% One such representation is marching cubes~\cite{marching-cubes}, where voxels are defined as density functions to a surface.
% This can create smooth surfaces from volume data, which has its uses in medical imaging~\cite{marching-cubes} and 3D-terrain~\cite{marching-cubes-terrain}.

As voxels are just a way to represent volumetric data, they have been used in a wide range of applications, such as global illumination~\cite{crassin-VCT}, medical imaging~\cite{marching-cubes,voxel-medicin} and collision culling~\cite{voxel-collision}.
These applications either store a voxel representation of models (especially true for medical imaging) or need to convert models into voxels.
This process is called \textit{voxelization} and has been widely studied in the past.
Some methods for voxelizations include triangle-box intersection~\cite{SAT-voxelization}, rasterization~\cite{octree-voxelization} and depth buffer~\cite{depth-buffer}.

Recently, a study was published by \call{scanline-voxelization} proposing a new method to voxelize a model.
The basics of the method is to voxelize the model by performing line voxelization at different stages.
This method is the basis of this thesis and will further be called the \textit{optimal scanline}. 
% The authors used both real and integer line voxelization based on~\call{voxeltraversal}.
% These two will further be called \textit{Real Line Voxelization} (RLV) and \textit{Integer Line Voxelization} (ILV).
% They mentioned ILV was slightly more efficient ($\sim$3\%), but the actual data was not published.

\newpage

\section{Aim}
This thesis aims to do an investigation of how the line voxelization algorithm affects the performance of the optimal scanline.
This includes both floating-point and integer line voxelizations.

The thesis also investigates the approximation error caused by using the integer versions of the algorithm.
This was done in \cite{scanline-voxelization}, but the authors only compared the voxel count.
Which means if a voxel is moved somewhere else, it would not produce any error.
As such, this thesis aims to measure this error in another metric, which can describe those errors.

\section{Research Questions}
With the aim of the thesis defined, two research questions are formed as a baseline of the thesis.
These are presented below:
\begin{enumerate}
  \item \label{que:linealg} Which line voxelization algorithm performs best for the optimal scanline?
  \item \label{que:erroraprox} How great is the approximation error of the integer versions of the optimal scanline?
  % \item \label{que:combine} Can the different line voxelization be combined to improve performance and accuracy?
\end{enumerate}
A better performance in this case is defined as how fast the execution of the voxelization is.
It is not a metric of the error or the memory usage of the algorithms.

\section{Delimitations}
The implementation of the thesis ran on \textit{Amazon Web Services} (AWS), on a computer running Ubuntu 18.04 with an NVIDIA Tesla T4 graphics card with driver version 440.59.

The rendering of the voxelization was done using OpenGL.
As the focus was not on supporting older devices, the OpenGL version 4.6 was used.

The computing language of choice was CUDA, as such, other languages were not considered for the implementation.
Again, as there is no need to support older devices, CUDA 10.2 was used.

The line algorithms that were evaluated were limited to a voxel traversal algorithm by \call{voxeltraversal}, its integer version and Bresenham.
As Bresenham is predominantly a 2D line drawing algorithm, it was extended to 3D using a method proposed by \call{3d-bresenham}.

The evaluations were also limited to three different models and voxel grid resolutions between 128-2048.

\newpage

\section{Mindroad}
This master thesis was conducted on behalf of MindRoad. 
MindRoad is a software company which specializes in embedded systems, web development and mobile applications. 
They also provide courses in software development, such as C++, Python, GPU-programming, Linux and driver development. 
